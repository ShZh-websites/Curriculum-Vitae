\documentclass{common}
\usepackage{xeCJK}
\setCJKmainfont{思源宋体}
\setCJKsansfont{思源黑体}

% \cfoot{\color{gray} 此文件的MD5值为\{\{TBD\}\}.}

\begin{document}

\name{沈之豪}
\vspace{0.3ex}
\basicInfo{
    \email{\underline{i@shzh.me}} |
    \homepage[\underline{shzh.wiki}]{https://shzh.wiki} |
    \github[\underline{Sh-Zh-7}]{https://github.com/Sh-Zh-7} |
    \linkedin[\underline{LinkedIn}]{https://www.linkedin.com/in/imshzh7/}
}

\vspace{0.5ex}

\section{\faGraduationCap\ 教育}
\educationsubsection
    {武汉大学 (WHU)}{GPA: 3.65/4.0}{2018 -- 2022}
    {计算机学院 | 计算机科学与技术(CS)专业的\textit{本科生}。}{中国\ 武汉}
\educationsubsectionwithoutgpa
    {清华大学 (THU)}{2023 -- 2026 (预期)}
    {软件学院 | 软件工程(SE)专业的\textit{硕士研究生}。}{中国\ 北京}

\vspace{0.85ex}

\section{\faCubes\ 项目}
\projectsubsection
    {\textbf{sznn}}
    {https://github.com/ShZh-libraries/sznn}{8}
    {2022年2月 -- 2022年4月}
\role{语言: Rust/Typescript}{毕业设计}
运行在 Web 上的高性能神经网络推理框架。
\begin{itemize}
    \item 支持纯 Javascript 的后端,可移植性强。
    \item 通过 Rust 生成 WASM ,并用其 SIMD 和多线程新特性加速。
    \item 采用 WebGPU 在 GPU 上进行运算,进一步提高了吞吐量。
\end{itemize}
\projectsubsectionwithoutstar
    {\textbf{ShZh Websites}}
    {https://github.com/ShZh-websites}
    {2021年9月 -- 现今}
\role{语言: Svelte/ReScript/Tex}{个人项目}
我个人网站的集合,涵盖了我的个人主页,博客,日志等等。
\begin{itemize}
    \item 针对用户体验高度优化, Lighthouse 高分保障。
    \item 使用了 Alpine, Tailwind 和 Gatsby 等现代 Web 技术。
    \item 拥有适配各种屏幕大小的响应式设计和自适应的黑暗模式。
\end{itemize}
\projectsubsectionwithoutstar
    {\textbf{Robot V0}}
    {https://github.com/Sh-Zh-7/Robot-V0}
    {2022年5月 -- 现今}
\role{语言: Kotlin}{个人项目}
我的多功能 Q 群机器人,由 OneBot 协议和 go-cqhttp 强力驱动。
\begin{itemize}
    \item 嗅探相关链接并用 OpenGraph 可视化展示。
    \item 基于 RSSHub 实现高度可扩展的主流社交媒体的订阅。
    \item 高自由度历史记录查询来替代QQ仅依赖内容的原生查询功能。
    \item 移植 Twitter 热门的 makeitaquote 和 progress-of-year 等功能。
\end{itemize}

\vspace{0.85ex}

\section{\faTrophy\ 获奖}
\award
    {\textbf{- 中国软件杯}, 三等奖, \textit{队长}}{2020年8月}
    {基于计算机视觉的交通场景智能应用。 
        {\href{https://github.com/Sh-Zh-7/intelligent-transportation-system}{\underline{\faGithub}} \faStarO\ 37}
    }
\vspace{0.5ex}
\award
    {\textbf{- APMCM}, 二等奖}{2019年11月}
    {关于区域经济活力及其影响因素的分析和决策。}

\vspace{0.85ex}

\section{\faInfo\ 其他}
\begin{itemize}[parsep=0.5ex]
    \item \textbf{爱好}: 体系结构(计算|存储)
    \item \textbf{英语}: CET-6: 463 | CET-4: 556
    \item \textbf{Linktree}: \href{https://linktr.ee/ShZh7}{\underline{https://linktr.ee/ShZh7}}
\end{itemize}

\end{document}
