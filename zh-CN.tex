\documentclass{common}
\usepackage{xeCJK}
\setCJKmainfont{思源宋体}
\setCJKsansfont{思源黑体}

\cfoot{\color{gray} 此文件的MD5值为\{\{TBD\}\}.}

\begin{document}

\name{沈之豪}
\basicInfo{
    \email{\underline{i@shzh.me}} |
    \homepage[\underline{shzh.wiki}]{https://shzh.wiki} |
    \github[\underline{Sh-Zh-7}]{https://github.com/Sh-Zh-7} |
    \linkedin[\underline{LinkedIn}]{https://www.linkedin.com/in/imshzh7/}
}

\section{\faGraduationCap\ 教育}
\educationsubsection
    {武汉大学 (WHU)}{GPA: 3.64/4.0}{2018 -- 2022}
    {计算机科学与技术(CS)的\textit{本科生}。}{中国\ 武汉}

\section{\faCubes\ 项目}
\projectsubsection
    {\textbf{sznn}}
    {https://github.com/ShZh-libraries/sznn}{6}
    {2022年2月 -- 2022年4月}
\role{Rust/Typescript, 维护者}{毕业设计}
运行在Web上的高性能神经网络推理框架。
\begin{itemize}
    \item 支持纯Javascript的后端。
    \item 使用WASM的SIMD和多线程技术进行加速。
    \item 通过WebGPU技术极大地提高了吞吐量。
\end{itemize}
\projectsubsection
    {\textbf{ShZh Playground}}
    {https://github.com/ShZh-Playground}{3}
    {2020年8月 -- 现今}
\role{C++/Rust/Java 等, 维护者}{个人项目}
我个人工具和轮子的集合,不定时更新。
\begin{itemize}
    \item Tiny MIPS CPU 和 tiny-OS 是本科课程设计的要求。
    \item 数据库,编译器还有其他项目都是在空闲时间基于兴趣完成的。
    \item 还有其他类似于 react-like 库这样的高级项目。
\end{itemize}
\projectsubsection
    {\textbf{ShZh Websites}}
    {https://github.com/ShZh-websites}{1}
    {2021年9月 -- 现今}
\role{Svelte/ReScript/Tex, 维护者}{个人项目}
我个人网站的集合,涵盖了我的个人主页,博客,日志等等。
\begin{itemize}
    \item 使用了 Alpine,Tailwind 和 Gatsby 等现代Web技术。
    \item 针对用户体验高度优化,有 Lighthouse 分数保障。
    \item 灵活的响应式设计,不仅适配了移动端,还有黑暗模式。
\end{itemize}

\section{\faTrophy\ awards}
\award
    {\textbf{- 中国软件杯}, 三等奖, \textit{队长}}{2020年8月}
    {基于计算机视觉的交通场景智能应用。 
        {\href{https://github.com/Sh-Zh-7/intelligent-transportation-system}{\underline{\faGithub}} \faStarO\ 30}
    }
\award
    {\textbf{- APMCM}, 二等奖}{2019年11月}
    {关于区域经济活力及其影响因素的分析和决策。}

\section{\faCogs\ 技能}
\begin{itemize}[parsep=0.5ex]
    \item \textbf{人工智能}: 机器学习, 强化学习
    \item \textbf{高性能计算}: C/C++/Rust, ARM/x86 SIMD, OpenMP, CUDA, MPI
    \item \textbf{数据库}: MySQL, Redis/memcached, boltDB, levelDB, DuckDB, .etc.
    \item \textbf{Web后端}: Java/Go/Kotlin, Sprint Boot/Spring Cloud, .etc.
    \item \textbf{Web前端}: H5/C3, JS/WASM, TypeScript/ReScript, Vue/React/Angular/Svelte
\end{itemize}

\section{\faInfo\ 其他}
\begin{itemize}[parsep=0.5ex]
    \item \textbf{普通话}: 母语
    \item \textbf{英语}: CET-6: 463 | CET-4: 556
    \item \textbf{Linktree}: \href{https://linktr.ee/ShZh7}{\underline{https://linktr.ee/ShZh7}}
\end{itemize}

\end{document}
